%!TEX ROOT=geometria1.tex

\section{Spazio vettoriale}%
\label{sec:spazio_vettoriale}

\begin{Def}{Spazio vettoriale}
  Un insieme $V$ si definisce spazio vettoriale sul campo $\mathbb{K}$ se sono definite
  su $V$ due operazioni
  \begin{enumerate}
    \item \textbf{Somma} definita come
      \begin{align*}
        +:\,&V\times V\to V\\
            &(\vec{x},\vec{y})\mapsto\vec{x}+\vec{y}
      \end{align*}
      rispetto alla quale $(V,\,+)$ ha la struttura di gruppo commutativo. Ovvero
      \begin{enumerate}
        \item $\vec{x}+\vec{y} = \vec{y}+\vec{x}$
        \item $(\vec{x}+\vec{y})+\vec{z} = \vec{x} + (\vec{y}+\vec{z})$
        \item $\exists\vec{0}\in V\suchthat \vec{x}+\vec{0}=\vec{x}$ e si definisce
          $\vec{0}$ vettore nullo.
        \item $\forall\vec{x}\in V\,\exists\vec{x}\in
          V\suchthat\vec{x}+(-\vec{x})=\vec{0}$ e si definisce opposto.
      \end{enumerate}
    \item \textbf{Prodotto} definito per uno scalare
      \begin{align*}
        &\mathbb{K}\times V\to V\\
        &(\lambda,\vec{x})\mapsto \lambda\vec{x}
      \end{align*}
      e si ha che
      \begin{enumerate}
        \item $\lambda(\vec{x}+\vec{y}) = \lambda\vec{x}+\lambda\vec{y}$
        \item $(\lambda+\mu)\vec{x} = \lambda\vec{x}+\mu\vec{x}$
        \item $(\lambda\mu)\vec{x} = \lambda(\mu\vec{x})$
        \item $1\vec{x} = \vec{x}$
      \end{enumerate}
  \end{enumerate}
\end{Def}

\begin{Def}{Eleementi dello spazio}
  Gli elementi di $V$ sono detti vettori, quelli di $\mathbb{K}$ scalari.
\end{Def}

\begin{Def}{Campo}
  Un campo è un insieme i cui elementi sono detti numeri, che contiene $0$ e $1$ e ha
  due operazioni $+$ e $\cdot$ che verificano
  \begin{multicols}{2}
    \begin{enumerate}
      \item $\alpha+\beta = \beta+\alpha$
      \item $\alpha+(\beta+\gamma) = (\alpha+\beta)+\gamma$
      \item $\alpha+0 = \alpha$
      \item $\alpha+(-\alpha)=0$
      \columnbreak%
      \item $\alpha\beta = \beta\alpha$
      \item $(\alpha\beta)\gamma = \alpha(\beta\gamma)$
      \item $1\alpha = \alpha$
      \item $\alpha\alpha^{-1}=1$ se $\alpha\neq0$
      \item $(\alpha+\beta)\gamma = \alpha\gamma+\beta\gamma$
    \end{enumerate}
  \end{multicols}
\end{Def}

\subsection{Spazi particolari}%
\label{sub:spazi_particolari}

In generale $\mathbb{R}^n$ è uno spazio vettoriale, così come anche in generale
$\mathbb{K}^n$. Infatti si ha che
\begin{equation*}
  (x_1,\ldots,x_2)+(y_1,\ldots,y_n) = (x_1+y_1,\ldots,x_n+y_n)
\end{equation*}
e
\begin{equation*}
  \lambda(x_1,\ldots,x_n) = (\lambda x_1,\ldots,\lambda x_n)
\end{equation*}
In generale anche $\mathbb{K}^{m,n}$ è uno spazio vettoriale (e quindi anche
$\mathbb{R}^{m,n}$).\\
Il più piccolo spazio vettoriale è quello composto dal solo vettore nullo, ovvero
$\{\vec{0}\}$.\
Un caso particolare è lo spazio dei polinomi reali in $x$, denotato come $\mathbb{R}[x]$
che è
\begin{equation*}
  \mathbb{R}[x]\bydef \left\{ a_0+a_1x+\cdots+a_n x^n\Setsuchthat
  n\in\mathbb{N},\,a_i\in\mathbb{R},\,i=0,\ldots,n \right\}
\end{equation*}
È anche interessante il caso in cui si consideri l'insieme
\begin{equation*}
  \mathscr{F} = \left\{ f:\,\mathbb{R}\to\mathbb{R}\qq{funzione} \right\}
\end{equation*}
in quanto anche questo è uno spazio vettoriale infatti
\begin{equation*}
  (f+g)(x)\bydef f(x)+g(x)
\end{equation*}
e
\begin{equation*}
  (\lambda f)(x)\bydef \lambda f(x)
\end{equation*}

\subsection{Proprietà formali}%
\label{sub:proprieta_formali}

In un campo vettoriale su $\mathbb{K}$ valgono le seguenti proprietà
\begin{enumerate}
  \item Vettore nullo unico
  \item Opposto unico
  \item Se per $\vec{x}$, $\vec{y}$, $\vec{z}$ si ha $\vec{x}+\vec{y}=\vec{x}+\vec{z}$
    allora $\vec{y}=\vec{z}$
  \item Solo su $\mathbb{R}$ vale che $\lambda\vec{x}=\vec{0}$ con
    $\lambda\in\mathbb{R}$ allora $\lambda=0\lor\vec{x}=\vec{0}$
  \item $(-1)\vec{x} = -\vec{x}$
\end{enumerate}
