%!TEX ROOT=geometria1.tex

\section{Equazioni lineari}%
\label{sec:equazioni_lineari}

\begin{Def}{Equazione lineare}
  Un'equazione lineare nelle incognite $x_1,x_2,\ldots,x_n$ è un'espressione del tipo
  \begin{equation}\label{eq:lineare}
    a_1x_1+a_2x_2+\cdots+a_n x_n=b
  \end{equation}
  dove $a,b\in\mathbb{R}, i=1,\ldots,n$.\\
  $a_i$ sono detti coefficienti, $b$ è detto termine noto.\\
  Scritta in forma matriciale
  \begin{equation*}
    \begin{pmatrix}
      a_1&a_2&a_3&\cdots&a_n
    \end{pmatrix}
    \begin{pmatrix}
      x_1\\x_2\\\vdots\\x_n
    \end{pmatrix}
    = b
  \end{equation*}
\end{Def}

\begin{Def}{Soluzione dell'equazione lineare}
  Una soluzione dell'equazione lineare~\eqref{eq:lineare} è una $n$-upla di numeri reali
  $(\tilde{x_1},\tilde{x_2},\ldots,\tilde{x_n})$ che sostituiti nell'equazione, la
  verifica.
\end{Def}

\begin{Def}{Equazione lineare omogenea}
  L'equazione~\eqref{eq:lineare} si dice omogenea se $b=0$.
\end{Def}

\begin{SubDef}{Soluzione particolare}
  La $n$-upla $(0,0,\ldots,0)$ è soluzione dell'equazione omogenea.
\end{SubDef}

\begin{SubDef}{Soluzione particolare}
  Se $(\tilde{x_1},\ldots,\tilde{x_n})$ è soluzione, lo è anche
  $(t\tilde{x_1},\ldots,t\tilde{x_n})$.
\end{SubDef}
