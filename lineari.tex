%!TEX ROOT=geometria1.tex

\section{Equazioni e sistemi lineari}%
\label{sec:equazioni_e_sistemi_lineari}

\subsection{Equazioni lineari}%
\label{sub:equazioni_lineari}

\begin{Def}{Equazione lineare}
  Un'equazione lineare nelle incognite $x_1,x_2,\ldots,x_n$ è un'espressione del tipo
  \begin{equation}\label{eq:lineare_equazione}
    a_1x_1+a_2x_2+\cdots+a_n x_n=b
  \end{equation}
  dove $a,b\in\mathbb{R}, i=1,\ldots,n$.\\
  $a_i$ sono detti coefficienti, $b$ è detto termine noto.\\
  Scritta in forma matriciale
  \begin{equation*}
    \begin{pmatrix}
      a_1&a_2&a_3&\cdots&a_n
    \end{pmatrix}
    \begin{pmatrix}
      x_1\\x_2\\\vdots\\x_n
    \end{pmatrix}
    = b
  \end{equation*}
\end{Def}

\begin{Def}{Soluzione dell'equazione lineare}
  Una soluzione dell'equazione lineare~\eqref{eq:lineare_equazione} è una $n$-upla di
  numeri reali
  $(\tilde{x}_1,\tilde{x}_2,\ldots,\tilde{x}_n)$ che sostituiti
  nell'equazione, la verifica.
\end{Def}

\begin{Def}{Equazione lineare omogenea}
  L'equazione~\eqref{eq:lineare_equazione} si dice omogenea se $b=0$.
\end{Def}

\begin{SubDef}{Soluzione particolare}
  La $n$-upla $(0,0,\ldots,0)$ è soluzione dell'equazione omogenea.
\end{SubDef}

\begin{SubDef}{Soluzione particolare}\label{def:lineare_equazione_soluzione_particolare_2}
  Se $(\tilde{x}_1,\ldots,\tilde{x}_n)$ è soluzione, lo è anche
  $(t\tilde{x}_1,\ldots,t\tilde{x}_n)$.
\end{SubDef}

\subsection{Sistemi lineari}%
\label{sub:sistemi_lineari}

\begin{Def}{Sistema lineare}
  Un sistema lineare di $m$ equazioni e $n$ incognite $x_1,\ldots,x_n$ è un instieme di
  equazioni lineari del tipo
  \begin{equation}\label{eq:lineare_sistema}
    \begin{cases}
      a_{11}x_1+\cdots+a_{1m}x_n &= b_1\\
      \vdots &= \vdots\\
      a_{m1}x_1+\cdots+a_{mn}x_n &= b_m
    \end{cases}
  \end{equation}
  $a_{ij}$ si dicono coefficienti (con $i=1,\ldots,n,\;j=1,\ldots,m$).\\
  $b_{i}$ si dicono termini noti.\\
  Scritto in forma matriciale
  \begin{equation*}
    AX=B
  \end{equation*}
  in cui
  \begin{alignat*}{2}
    A &=
    \begin{pmatrix}
      a_{11} & \cdots & a_{1m}\\
      \vdots & \ddots & \vdots\\
      a_{m1} & \cdots & a_{mn}
    \end{pmatrix}\in\mathbb{R}^{m,n} &\qquad &\text{Matrice dei coefficienti}\\
    X &=
    \begin{pmatrix}
      x_1\\\vdots\\x_n
    \end{pmatrix}\in\mathbb{R}^{n,1} & B &=
    \begin{pmatrix}
      b_1\\\vdots\\b_m
    \end{pmatrix}\in\mathbb{R}^{m,1}
  \end{alignat*}
\end{Def}

\begin{Def}{Matrice completa}
  Si definisce matrice completa
  \begin{equation*}
    (A\vert B)=
    \begin{pmatrix}[ccc|c]
      a_{11} & \cdots & a_{1m} & b_1\\
      \vdots & \ddots & \vdots & \vdots\\
      a_{m1} & \cdots & a_{mn} & b_m
    \end{pmatrix}
  \end{equation*}
  Ciascuna riga si indica con $R_i$.
\end{Def}

\begin{Def}{Sistema lineare omogeneo}
  Un sistema lineare è omogeneo se $b_j=0\;\forall j=1,\ldots,m$, ovvero
  \begin{equation*}
    AX=O
  \end{equation*}
\end{Def}

\begin{Def}{Soluzione del sistema lineare}
  Soluzione del sistema lineare~\eqref{eq:lineare_sistema} è una $n$-upla di numeri
  reali $(\tilde{x}_1,\ldots,\tilde{x}_1)$ che sostituia nelle ingognite
  verifica tutte le equazioni.
\end{Def}

\begin{SubDef}{Soluzioni di un sistema omogeneo}
  Se un sistema lineare è omogeneo, allora $(0,\ldots,0)$ è una sua soluzione. Si
  conclude quindi che un sistema lineare omogeneo è sempre compatibile.
\end{SubDef}

\begin{Def}{Sistema compatibile}
  Un sistema lineare si dice compatibile se ammette soluzioni, incompatibile altrimenti.
\end{Def}

\begin{Def}{Sistema equivalente}
  Un sistema si dice equivalente ad un altro se ammette le stesse soluzioni.
\end{Def}

\subsubsection{Metodo di riduzione di Gauss}%
\label{ssub:metodo_di_riduzione_di_gauss}

Il metodo di riduzione di Gauss permette di semplificare un sistema lineare in uno
equivalente.

\begin{Thm}{Operazioni elementari di riduzione per righe}
  Eseguendo un numero finito di volte le tre operazioni
  \begin{enumerate}
    \item\label{thm:lineare_riduzione_1} Scambiare due equazioni
    \item\label{thm:lineare_riduzione_2} Moltiplicare per un numero reale diverso da
      $0$
    \item\label{thm:lineare_riduzione_3} Sostituire ad un'equazione la somma di se
      stessa con un'altra equazione moltiplicata per un qualsiasi numero reale
  \end{enumerate}
  si ottiene un sistema lineare equivalente.
\end{Thm}

\begin{proof}
  Dimostrare~\ref{thm:lineare_riduzione_1} è ovvio, in quanto le equazioni non si
  modificano.\\
  Il punto~\ref{thm:lineare_riduzione_2} invece deve essere dimostrato che se una
  $n$-upla è soluzione di un sistema, lo è anche dell'altro e viceversa. Si ha quindi
  \begin{equation*}
    \begin{cases}
      a_{11}x_1+\ldots a_{1n}x_n &= b_1\\
      \vdots &= \vdots\\
      a_{m1}x_1+\ldots a_{mn}x_m &= b_m
    \end{cases} \implies
    \begin{cases}
      \lambda(a_{11}x_1+\ldots a_{1n}x_n) &= \lambda b_1\\
      \vdots &= \vdots\\
      a_{m1}x_1+\ldots a_{mn}x_m &= b_m
    \end{cases}
  \end{equation*}
  Per~\autoref{def:lineare_equazione_soluzione_particolare_2} si ha che la seconda
  equazione ha le stesse soluzioni della prima.
  \begin{equation*}
    \begin{cases}
      \lambda(a_{11}x_1+\ldots a_{1n}x_n) &= \lambda b_1\\
      \vdots &= \vdots\\
      a_{m1}x_1+\ldots a_{mn}x_m &= b_m
    \end{cases} \implies
    \begin{cases}
      a_{11}x_1+\ldots a_{1n}x_n &= b_1\\
      \vdots &= \vdots\\
      a_{m1}x_1+\ldots a_{mn}x_m &= b_m
    \end{cases}
  \end{equation*}
  dividendo per $\lambda\neq0$.
  Per~\autoref{def:lineare_equazione_soluzione_particolare_2} si ha che hanno le stesse
  soluzioni.\\
  Per il punto~\ref{thm:lineare_riduzione_3} si procede analogamente al
  punto~\ref{thm:lineare_riduzione_2}.
\end{proof}

Dal punto di vista matriciale, le trasformazioni si applicano nei seguenti modi
\begin{align*}
  R_i &\leftrightarrow R_j\\
  R_i &\leftrightarrow \lambda R_i\quad\lambda\neq0\\
  R_i &\leftrightarrow R_i+\lambda R_j\quad\lambda\in\mathbb{R},\;j\neq i
\end{align*}
Eseguire queste operazioni un numero finito di volte significa trasformare $(A\vert B)$
in $(\widetilde{A}\vert\widetilde{B})$ in modo che ogni riga di $\widetilde{A}$ non
nulla esista un elemento non nullo sotto il quale sono tutti $0$.

\begin{Def}{Sistema ridotto}
  Un sistema lineare è ridotto se è ridotta $A$.
\end{Def}

\begin{Thm}{Teorema di Rouché-Capelli}\label{thm:lineare_rouche-capelli}
  Un sistema lineare di $m$ equazioni e $n$ incognite
  \begin{equation*}
    AX = B
  \end{equation*}
  è compatibile se e solo se
  \begin{equation*}
    \rank(A) = \rank(A\vert B)
  \end{equation*}
\end{Thm}

In particolare si ha che se $\rank(A)=\rank(A\vert B)=n$ la soluzione è unica. Se invece
$\rank(A)=\rank(B)=k<n$ ci sono infinite soluzioni che dipendono da $n-k$ variabili.
Quindi ci sono $\infty^{n-k}$ soluzioni.

\begin{SubThm}{Teorema di Rouché-Capelli per un sistema lineare omogeneo}
  Un sistema lineare omogeneo di $m$ equazioni e $n$ incognite
  \begin{equation*}
    AX = O
  \end{equation*}
  è sempre compatibile. Se
  \begin{equation*}
    \rank(A) = n
  \end{equation*}
  esiste un'unica soluzione che è quella nulla. Se
  \begin{equation*}
    \rank(A) = k < n
  \end{equation*}
  il sistema ammette $\infty^{n-k}$ soluzioni.
\end{SubThm}

Si noti che se $AX=B$ ha un'unica soluzione e utilizzando il metodo di riduzione di
Gauss-Jordan si può arrivare ad una matrice ridotta a scala del tipo
\begin{equation*}
  \begin{pmatrix}[cccc|c]
    1 & 0 & \cdots & 0 & \tilde{b}_1\\
    0 & 1 & \cdots & \vdots & \vdots\\
    \vdots & \cdots & \ddots & \vdots & \vdots\\
    0 & \cdots & \cdots & 1 & \tilde{b}_n
  \end{pmatrix}
\end{equation*}
in cui si ha che $A=I$.


\subsection{Equazioni matriciali}%
\label{sub:equazioni_matriciali}

\begin{Def}{Equazione matriciale}
  Un'equazione matriciale è un'equazione del tipo
  \begin{equation*}
    AX = B
  \end{equation*}
  con $A\in\mathbb{R}^{m,n}$, $X\in\mathbb{R}^{n,p}$, $B\in\mathbb{R}^{m,p}$.
\end{Def}

\begin{SubDef}{Casi particolari}
  Se $p=1$, si ha un sistema lineare.\\
  Se $AX=I$, si ha che $X$ è l'inversa di $A$.\\
  Se si ha $YC=D$, si può ricondurre in modo che $\transp(YC)=\transp C\iff\transp
  C\transp Y = \transp D$.
\end{SubDef}

Se ad esempio si pensa di scrivere $X$ come matrice colonna di $n$-uple, del tipo
\begin{equation*}
  X =
  \begin{pmatrix}
    X_1\\
    \vdots\\
    X_n
  \end{pmatrix}
\end{equation*}
e la stessa cosa per $B$
\begin{equation*}
  B =
  \begin{pmatrix}
    B_1\\
    \vdots\\
    B_n
  \end{pmatrix}
\end{equation*}
si può scrivere l'equazione matriciale come sistema
\begin{equation}\label{eq:lineare_matriciale}
  AX = B \iff
    \begin{cases}
      a_{11}x_1 + \cdots + a_{1m}x_1 &= B_1\\
      \vdots & \vdots\\
      a_{n1}x_1 + \cdots + a_{nm}x_n &= B_n
    \end{cases}
\end{equation}
Si può notare come~\eqref{eq:lineare_matriciale} sia equivalente ad un sistema lineare
di $pn$  incognite $x_{ij}$.
