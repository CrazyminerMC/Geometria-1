%!TEX ROOT=geometria1.tex

\section{Esempi}%
\label{sec:esempi}

Qui verranno riportati alcuni esempi di teoremi, proprietà o semplici esercizi che
mostrano un'applicazione pratica della teoria.

\subsection{\nameref{sec:matrici}}%

\subsection{\nameref{sec:equazioni_e_sistemi_lineari}}%

\subsubsection{Numero di soluzioni di un sistema}%
\label{ssub:numero_di_soluzioni_di_un_sistema}

\paragraph{Esempio 1}%
\label{par:esempio_1}

\textbf{Si discuta il numero di soluzioni del seguente sistema}
\begin{equation*}
  \begin{cases}
    x_1+x_2-x_3 = 1\\
    2x_1+2x_2+x_3=0\\
    x_1+x_2+2x_3=-1
  \end{cases}
\end{equation*}

Si scrive subito la matrice completa associata
\begin{equation*}
  (A\vert B) =
  \begin{pmatrix}[ccc|c]
    1 & 1 & -1 & 1\\
    2 & 2 & 1 & 0\\
    1 & 1 & 2 & -1
  \end{pmatrix}
\end{equation*}
Ora dobbiamo cercare di ridurla per righe in modo da oter determinare il numero di
soluzioni.
\begin{equation*}
  (A\vert B)\xrightarrow{\substack{R_2\to R_2-2R_1\\R_3\to R_3-R1}}
  \begin{pmatrix}[ccc|c]
    1 & 1 & -1 & 1\\
    0 & 0 & 3 & -2\\
    0 & 0 & 3 & -2
  \end{pmatrix}\xrightarrow{\substack{R_3\to R_3-R_2}}
  \begin{pmatrix}[ccc|c]
    1 & 1 & -1 & 1\\
    0 & 0 & 3 & -2\\
    0 & 0 & 0 & 0
  \end{pmatrix}
\end{equation*}
A questo punto abbiamo ridotto per righe questa matrice. Possiamo tornare al sistema
\begin{equation*}
  \begin{cases}
    x_1+x_2-x_3=1\\
    3x_3=-2
  \end{cases}
\end{equation*}
Questo sistema è in due equazioni ma tre incognite, questo significa che se due sono
fissate, una è libera di modificarsi. Ovvero ci sono $\infty^1$ soluzioni.

\subsubsection{Uso del teorema di Rouché-Capelli}%
\label{sub:uso_del_teorema_di_rouche_capelli}

\paragraph{Esempio 1}%
\textbf{Discutere al variare di $h,k\in\mathbb{R}$ il sistema}
\begin{equation*}
  \begin{cases}
    kx + y + z = 1\\
    x + ky + z = 1\\
    x + y + kz = h
  \end{cases}
\end{equation*}
Si può riscrivere il sistema in forma matriciale con la matrice completa
\begin{equation*}
  (A\vert B) =
    \begin{pmatrix}[ccc|c]
    k & 1 & 1 & 1\\
    1 & k & 1 & 1\\
    1 & 1 & k & h
    \end{pmatrix}
\end{equation*}
Possiamo ora cercare di ridurre la matrice
\begin{equation*}
    \begin{pmatrix}[ccc|c]
    k & 1 & 1 & 1\\
    1 & k & 1 & 1\\
    1 & 1 & k & h
    \end{pmatrix}
  \xrightarrow{R_1\leftrightarrow R_2}
    \begin{pmatrix}[ccc|c]
    1 & k & 1 & 1\\
    k & 1 & 1 & 1\\
    1 & 1 & k & h
    \end{pmatrix}
  \xrightarrow{\substack{R_2\to R_2-kR_1\\ R_3\to R_3-R_1}}
  \begin{pmatrix}[ccc|c]
    1 & k & 1 & 1\\
    0 & 1-k & 1-k & 1-k\\
    0 & 1-k & 1-k & h-1
    \end{pmatrix}
\end{equation*}
A questo punto si distinguono due casi, se $1-k = 0$ o $1-k\neq0$.\\
Se $1-k=0\implies k=1$, sostituendo
\begin{equation*}
    \begin{pmatrix}[ccc|c]
    1 & 1 & 1 & 1\\
    0 & 0 & 0 & 0\\
    0 & 0 & 0 & h-1
    \end{pmatrix}
\end{equation*}
Per questa matrice ridotta per righe si ha $\rank(A)=1$. Per
il~\autoref{thm:lineare_rouche-capelli} si distinguono gli ultimi due casi per $h$. Se
$h=1$, allora il sistema è compatibile con $\infty^2$ soluzioni. Altrimenti non ci sono
soluzioni in quanto $\rank(A\vert B) = 2 \neq \rank(A)$.\\
Se $1-k\neq0\implies k\neq1$ i può dividere la seconda e terza riga per $\frac{1}{1-k}$
ottenendo
\begin{equation*}
    \begin{pmatrix}[ccc|c]
    1 & k & 1 & 1\\
    0 & 1+k & 1 & 1\\
    0 & 1 & -1 & \frac{h-1}{1-k}
    \end{pmatrix}
\end{equation*}
Andando a sommare $R_3$ con $R_2$, si ha
\begin{equation*}
    \begin{pmatrix}[ccc|c]
    1 & k & 1 & 1\\
    0 & 1+k & 1 & 1\\
    0 & 2+k & 0 & \frac{h-1}{1-k}+1
    \end{pmatrix}
\end{equation*}
A questo punto abbiamo due casi: $k=-2$ e non. Immediatamente si vede che $k\neq-2$, si
ha che $\rank(A\vert B) = 3 \rank(A)$ e quindi la soluzione è unica per il
teorema~\autoref{thm:lineare_rouche-capelli}.\\
Se invece $k=-2$ si può riscrivere
\begin{equation*}
    \begin{pmatrix}[ccc|c]
      1 & -2 & 1 & 1\\
      0 & -1 & 1 & 1\\
      0 & 0 & 0 & \frac{h+2}{3}
    \end{pmatrix}
\end{equation*}
Si distinguono due casi a seconda di $h$. Se $h=-2$ o meno. Si vede immediatamente che
se $h\neq-2$ si ha che $\rank(A\vert B)=3\neq2$ e quindi non ci sono soluzioni per
il~\autoref{thm:lineare_rouche-capelli}.\\
Se invece $h=-2$ si ha che $\rank(A) = \rank(A\vert B) = 2$ e quindi ci sono $\infty^1$
soluzioni.\\
Riassumendo
\begin{equation*}
  \begin{cases}
    k=1,
    \begin{cases}
      h=1\implies\infty^2\\
      h\neq 1\implies 0
    \end{cases}\\
    k\neq1,
    \begin{cases}
      k=-2\implies
      \begin{cases}
        h=-2\implies\infty^1\\
        h=2\implies 0
      \end{cases}\\
      k\neq-2\implies 1
    \end{cases}
  \end{cases}
\end{equation*}
