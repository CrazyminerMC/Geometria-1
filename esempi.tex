%!TEX ROOT=geometria1.tex

\section{Esempi}%
\label{sec:esempi}

Qui verranno riportati alcuni esempi di teoremi, proprietà o semplici esercizi che
mostrano un'applicazione pratica della teoria.

\subsection{\nameref{sec:matrici}}%

\subsection{\nameref{sec:equazioni_e_sistemi_lineari}}%

\subsection{Uso del teorema di Rouché-Capelli}%
\label{sub:uso_del_teorema_di_rouche_capelli}

\paragraph{Esempio 1}%
\textbf{Discutere al variare di $h,k\in\mathbb{R}$ il sistema}
\begin{equation*}
  \begin{cases}
    kx + y + z = 1\\
    x + ky + z = 1\\
    x + y + kz = h
  \end{cases}
\end{equation*}

Si può riscrivere il sistema in forma matriciale con la matrice completa
\begin{equation*}
  (A\vert B) =
  \left(
    \begin{array}{@{}ccc|c@{}}
    k & 1 & 1 & 1\\
    1 & k & 1 & 1\\
    1 & 1 & k & h
    \end{array}
  \right)
\end{equation*}
Possiamo ora cercare di ridurre la matrice
\begin{equation*}
  \left(
    \begin{array}{@{}ccc|c@{}}
    k & 1 & 1 & 1\\
    1 & k & 1 & 1\\
    1 & 1 & k & h
    \end{array}
  \right) \xrightarrow{R_1\leftrightarrow R_2}
  \left(
    \begin{array}{@{}ccc|c@{}}
    1 & k & 1 & 1\\
    k & 1 & 1 & 1\\
    1 & 1 & k & h
    \end{array}
  \right) \xrightarrow{\substack{R_2\to R_2-kR_1\\ R_3\to R_3-R_1}}
  \left(
    \begin{array}{@{}ccc|c@{}}
    1 & k & 1 & 1\\
    0 & 1-k & 1-k & 1-k\\
    0 & 1-k & 1-k & h-1
    \end{array}
  \right)
\end{equation*}
A questo punto si distinguono due casi, se $1-k = 0$ o $1-k\neq0$.\\
Se $1-k=0\implies k=1$, sostituendo
\begin{equation*}
  \left(
    \begin{array}{@{}ccc|c}
    1 & 1 & 1 & 1\\
    0 & 0 & 0 & 0\\
    0 & 0 & 0 & h-1
    \end{array}
  \right)
\end{equation*}
Per questa matrice ridotta per righe si ha $\rank(A)=1$. Per
il~\autoref{thm:lineare_rouche-capelli} si distinguono gli ultimi due casi per $h$. Se
$h=1$, allora il sistema è compatibile con $\infty^2$ soluzioni. Altrimenti non ci sono
soluzioni in quanto $\rank(A\vert B) = 2 \neq \rank(A)$.\\
Se $1-k\neq0\implies k\neq1$ i può dividere la seconda e terza riga per $\frac{1}{1-k}$
ottenendo
\begin{equation*}
  \left(
    \begin{array}{@{}ccc|c@{}}
    1 & k & 1 & 1\\
    0 & 1+k & 1 & 1\\
    0 & 1 & -1 & \frac{h-1}{1-k}
    \end{array}
  \right)
\end{equation*}
Andando a sommare $R_3$ con $R_2$, si ha
\begin{equation*}
  \left(
    \begin{array}{@{}ccc|c@{}}
    1 & k & 1 & 1\\
    0 & 1+k & 1 & 1\\
    0 2+k & 0 & \frac{h-1}{1-k}+1
    \end{array}
  \right)
\end{equation*}
A questo punto abbiamo due casi: $k=-2$ e non. Immediatamente si vede che $k\neq-2$, si
ha che $\rank(A\vert B) = 3$ e quindi non ci sono soluzioni per il
teorema~\autoref{thm:lineare_rouche-capelli}.\\
Se invece $k=-2$ si può riscrivere
\begin{equation*}
  \left(
    \begin{array}{@{}ccc|c@{}}
      1 & -2 & 1 & 1\\
      0 & -1 & 1 & 1\\
      0 & 0 & 0 & \frac{h+2}{3}
    \end{array}
  \right)
\end{equation*}
Si distinguono due casi a seconda di $h$. Se $h=-2$ o meno. Si vede immediatamente che
se $h\neq-2$ si ha che $\rank(A\vert B)=3\neq2$ e quindi non ci sono soluzioni per
il~\autoref{thm:lineare_rouche-capelli}.\\
Se invece $h=-2$ si ha che $\rank(A) = \rank(A\vert B) = 2$ e quindi ci sono $\infty^1$
soluzioni.\\
Riassumendo
\begin{equation*}
  \begin{cases}
    k=1,
    \begin{cases}
      h=1\implies\infty^1\\
      h\neq 1\implies 0
    \end{cases}\\
    k\neq1,
    \begin{cases}
      h=-2\implies\infty^1\\
      h\neq-2\implies 0
    \end{cases}
  \end{cases}
\end{equation*}
